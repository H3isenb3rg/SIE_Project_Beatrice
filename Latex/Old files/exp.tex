\chapter{Computational Experiments}\label{ch:computational-experiments}
In this section we report the results of tests executed on the default Kernel Search applied to the MKP\@.
The main objective of these experiments is to determine the performance of the default Kernel Search,
both in terms of quality of the solution found and time required for the execution.


\section{Test Instances}
The test instances for the MKP were kindly provided to us by the authors of~\cite{mkp:2019}.\\
There are in total 2100 test instances, organized in
five directories of increasing complexity:
\begin{itemize}
    \item SMALL
    \item FK\_1
    \item FK\_2
    \item FK\_3
    \item FK\_4
\end{itemize}
SMALL contains 180 instances with \(m \in {10,20} \) and
\(n \in {20,40,60}\), while FK\_1, FK\_2, FK\_3, FK\_4 contain 480
instances each, designed with the aim of identifying critical ratios of
\(n/m\) that produce difficult instances.

The weights \(w_{j}\) are always uniformly distributed in \([\alpha, 1000]\), with \(\alpha = 1\) for the
SMALL instances and \(\alpha = 10\) for the FK instances.

Each instance belongs to one of the following four correlation classes:
\begin{itemize}
    \item \textit{uncorrelated}: profits \(p_{j}\) are uniformly distributed in \([\alpha,1000]\);
    \item \textit{weakly correlated}: For SMALL, \(p_{j} = 0.6 w_{j} + \theta_{j}\) , with \(\theta_{j}\) uniformly random in
    \([1, 400]\).
    For FK, the \(p_{j}\) values are uniformly distributed in \([w_{j} - 100, w_{j} + 100]\), such that \(p_{j} \geq 1\);
    \item \textit{strongly correlated}: For SMALL, \(p_{j} = w_{j} + 200\), for FK, \(p_{j} = w_{j} + 10\);
    \item \textit{subset-sum}: \(p_{j} = w_{j}\).
\end{itemize}

All the instances in SMALL and FK\_1 were already solved to the optimum.
Of those in the three remaining folders only a few were already solved:
the unsolved ones only had an upper and lower bound.

\subsection{Format of the Instances}
The instances are contained in plain \texttt{.txt} files.

An example of the structure of an instance is shown below:
\[
    \begin{split}
        &\begin{rcases*}
             3 \quad
        \end{rcases*}
        \rightarrow \text{integer that indicates the value of \textit{m (number of knapsacks)}.}\\
        &\begin{rcases*}
             7 \quad
        \end{rcases*}
        \rightarrow \text{integer that indicates the value of \textit{n (number of items)}.}\\
        &\begin{rcases*}
             908 \quad\\
             834\\
             675\\
        \end{rcases*}
        \rightarrow \text{integers that indicate the capacity } c_{i} \text{ of the knapsack i with }
        \space i=1,\dots,m.\\
        &\begin{rcases*}
             264    \quad 430 \quad\\
             606 \quad    945\\
             268 \quad    409\\
             619 \quad    591\\
             958 \quad    839\\
             972 \quad    818\\
             723 \quad    71\\
        \end{rcases*}
        \rightarrow
        \begin{rcases*}
            \text{First column shows the weight } w_{i} \quad \\
            \text{Second column shows the profit } p_{i}.
        \end{rcases*}
        \rightarrow
        \space i=1,\dots,n.
    \end{split}
\]

The execution times required to solve the instances to the optimum were not provided to us.
For a few of the instances we were able to find the exact solution
using a GUROBI solver with the following configuration:

\begin{itemize}
    \item \textit{Presolve}: 2 (aggressive presolve)
    \item \textit{Time limit}: 1 hour
    \item \textit{Threads}: 12
    \item \textit{MIPGap}: \num{1e-12}
\end{itemize}

Most of the instances however used all the available time without reaching the optimum.
The code and the results can be found at \url{https://github.com/Golino98/EsattoMKP}.

\subsection{Selected Instances}\label{subsec:inst}
To simplify the testing of the changes to the Kernel Search algorithm,
we only considered 40 instances: 6 solved to the optimum for each directory,
plus 10 that were not solved to the optimum.

Tables~\ref{tab:exact_opt} and~\ref{tab:exact_noopt} represent,
respectively, the optimum values of the 30 instances solved to
the optimum, and the upper and lower bound for the 10
that couldn't be solved.

\input{tables/exact_opt.tex}
\input{tables/exact_noopt.tex}


\section{Performance of the default Kernel Search}
Table ~\ref{tab:ker_per_val_abs_RC} outlines the result of tests executed on
the Kernel Search algorithm with the default configuration,
as outlined in ~\ref{ch:implementation}.
The tables include, for each instance:
\begin{itemize}
    \item Directory of the instance;
    \item Name of the instance;
    \item Solution found;
    \item Time elapsed;
    \item A boolean value that specifies if the time limit (120s) was reached.
\end{itemize}
\input{tables/kernelBase/KernelBuilderPercentage/VariableSorterByValueAndAbsoluteRC}