Dei 3 casi d'uso specificati viene presentata solamente la scheda di specifica dell'accesso alla lista 
degli eventi futuri perché gli altri due riguardano solamente la visualizzazione di una pagina statica:
\begin{itemize}
    \item \textbf{Accedi a Galleria} $\rightarrow$ Utente visualizza la pagina della galleria con foto e video di eventi passati
    \item \textbf{Accedi a Contatti e Membri del Team} $\rightarrow$ Utente visualizza i diversi canali attraverso i quali può contattare il team ed una lista di tutti i membri con foto e ruolo all'interno del team
\end{itemize}

\begin{table}[H]
    \begin{center}  
        \begin{tabular}{ | l | p{13cm} |} % you can change the dimension according to the spacing requirements  
        \hline
        \textbf{Titolo} & \textbf{Accedi a lista Eventi Futuri}\\ \hline  
           
        Scopo & Come gli Utenti Esterni non Registrati possono visualizzare e filtrare una lista ordinata dei prossimi eventi programmati ed i relativi dettagli\\ \hline  
           
        Pre-Condizione & Nessuna\\ \hline  
           
        Post-Condizione & Utente visualizza correttamente gli eventi (ordinati in base alla data) ed i dettagli eventualmente filtrando la lista in base alla venue desiderata o in base a del testo inserito nella search box\\ \hline  
           
        Workflow & Il caso d'uso prevede i seguenti passi
        \begin{itemize}
            \item Visualizza una versione limitata della lista (i prossimi 4/5 eventi) direttamente nella homepage oppure entra nella pagina dedicata
            \item Nella pagina dedicata l'utente visualizza la lista di tutti gli eventi ed imposta i filtri
            \item Lista viene aggiornata e utente visualizza eventi che rispettano i filtri imposti
        \end{itemize}\\  
        \hline
           
        \end{tabular}  
        \caption{Scheda di Specifica - Accesso alla lista degli eventi}
    \end{center}  
\end{table}